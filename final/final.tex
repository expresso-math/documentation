% THIS IS SIGPROC-SP.TEX - VERSION 3.1
% WORKS WITH V3.2SP OF ACM_PROC_ARTICLE-SP.CLS
% APRIL 2009

\documentclass{acm_proc_article-sp}

\begin{document}

\title{ Expresso: Typesetting Handwritten Mathematical Expressions on the Post-PC Tablet Computer }
\subtitle{ [Project Final Report] }

\numberofauthors{2} 

\author{
\alignauthor
Josef Lange\\
       \affaddr{University of Puget Sound}\\
       \affaddr{3929 N. 29th St.}\\
       \affaddr{Tacoma, WA 98407}\\
       \email{jlange@pugetsound.edu}
\alignauthor       
Daniel Guilak\\
       \affaddr{University of Puget Sound}\\
       \affaddr{3396 Wheelock Student Center}\\
       \affaddr{Tacoma, WA 98416-3396}\\
       \email{dguilak@pugetsound.edu}
}
       
\date{20 April, 2013}

\maketitle

\begin{abstract}
This paper sets forth to describe our process and products of fulfilling the capstone requirement of the Bachelor's Degree in Computer Science at the University of Puget Sound. Our project was to research, design, and implement a tablet-based software solution for the capture of handwritten mathematical expressions and compilation of its mathematical representation into an appropriate representative language (such as \LaTeX or MathML).

Subjects approached in our project include image processing, artificial intelligence, application development for the Apple iOS platform (particularly in the tablet form factor), server development, as well as the adherence to a highly disciplined development cycle. We will attempt to most accurately and most briefly explain any uncommon concepts described hereafter.
\end{abstract}

\section{Introduction}
The utility of the touch-based computer interaction has only been recently fully realized, with the nascence of the ``Post-PC'' tablet in its migration away from the standard desktop usage paradigm commonly associated with computing. The modern tablet focuses on media consumption and so-called ``basic'' computing. In today's reality, most ``Post-PC'' tablets are incredibly powerful both in hardware and in software, supporting complex and challenging computations including the decoding of video, image processing, interpretation of multiple inputs, managing several network connections, and displaying high-resolution two- and three-dimensional graphics. Past their technical capabilities, these small computers have found their way into near-ubiquity of use. 

The touch interface of a ``Post-PC'' tablet offers a unique tool to any consumer---the direct input of figures via touch---that has only previously been common to graphics and imaging professionals. Expresso sets out to take advantage of this new ubiquity in 
\section{Related Work}

\subsection{General Handwriting Recognition} 
Handwriting recognition is a complex problem, only a segment of which my project wishes to solve. Existing solutions to handwriting recognition exist, and some  perform at a useable level. Frequently, these systems require a period of ``training'' before any recognition can be achieved. Others use artificial intelligence to conclude what a sequence of handwriting is meant to be. For full-on text recognition, this is useful and almost required for any success. Commercial products exist, including but not limited to software built into Microsoft's Windows and Apple's Mac OS X Operating Systems. Additionally, many third-party companies have produced software for a similar purpose.

\subsection{Mathematical Expression Capture}
General handwriting recognition solutions are frequently overpowered for the niche of mathematical expressions, in which there are fewer possible logical constructs, most of which follow a fairly standard form. Because of this, the subset of things needing recognition, and possibilities for what they could be recognized as, is significantly smaller. Academic projects, including those by Matsakis\cite{matsakis_recognition_1999}, Smithies et. al.\cite{smithies_handwriting-based_1999}, and Levin\cite{levin_cellwriter:_2007} exhibit valid solutions for the desktop model of usage, though are not implemented in a way that is ultimately accessible for the common user.

\section{Proposed Solution}
This project will produce a feature-complete, usable piece of software for the Apple iOS platform. This software will have a singular function: to capture handwritten mathematical expressions via the digitizing screen, and communicate with a server which will compile the interpreted characters into the appropriate mathematical representational language (namely \LaTeX or MathML). 

\section{Motivation \& Importance}
Throughout their history, touch devices have attempted to attain status as a "universal tool" -- something able to function seamlessly in their users' lives. Humans communicate naturally with handwriting, and devices from Apple Newtons, to Palm Pilots, to Pocket PCs, and now to the likes of the Apple iPad, Amazon Kindle Fire, and Microsoft Surface have tried to integrate handwriting recognition with little success since software keyboards are more familiar and efficient to most users. However, we believe that the size and shape of tablet devices are reminiscent of ``slates" used in classrooms to scribble out mathematical equations.

Both authors at one point or another have been frustrated with a homework assignments requiring mathematical typesetting in \LaTeX.  We believe it would be great for a user to jot down an equation on an iPad or other tablet device and have the \LaTeX\ code appear on the document they are editing on their laptop or desktop computer.

This software could easily improve the classroom environment in all levels of schooling. Getting mathematical equations into type is a frustrating task for many teachers in the K-12 levels, and even frustrating for the higher-education professor and student. With this software, student and educators alike can easily convert their glyphic mathematical thoughts into usable typesetting language. This will hopefully open doors to clearer and more specific teaching and learning throughout all levels of education.

\section{Method \& Implementation}

\subsection{Overview}

\subsection{"Barista"}

\subsection{"Roaster", "Bean", \& "Grinder"}

\subsection{"Expresso (iOS Client)"}

\section{Evaluation}

\section{Discussion \& Analysis}

\section{Conclusion}

%\end{document}  % This is where a 'short' article might terminate

\bibliographystyle{abbrv}
\bibliography{final_citations}  
% You must have a proper ".bib" file
%  and remember to run:
% latex bibtex latex latex
% to resolve all references
%
% ACM needs 'a single self-contained file'!
%
%APPENDICES are optional
%\balancecolumns


\balancecolumns

% That's all folks!
\end{document}
